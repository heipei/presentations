% vim:et:sw=2:ts=2

\mode<presentation> {
  \usetheme{RWTHRZ}
}

\usepackage[german]{babel}
\usepackage[utf8]{inputenc}
\usepackage{times}
\usepackage[T1]{fontenc}
\usepackage[right]{eurosym}
\usepackage{graphicx}
\usepackage{hyperref}
\usepackage{color}
\definecolor{red}{rgb}{0.9,0.2,0.2}
\definecolor{green}{rgb}{0.2,0.5,0.2}
\definecolor{blue}{rgb}{0.2,0.2,0.5}
\definecolor{darkblue}{rgb}{0,0,.5}
\definecolor{lightgray}{rgb}{0.9,0.9,0.9}
\definecolor{gray}{rgb}{0.4,0.4,0.4}
\hypersetup{pdftex=true, colorlinks=true, breaklinks=true, linkcolor=darkblue, menucolor=darkblue, pagecolor=darkblue, urlcolor=darkblue}
\setbeamertemplate{navigation symbols}{}
\title[Einführung in Ruby on Rails]{Einführung in Ruby on Rails}
\author[Gilger, Lederhofer]{}
\date[\today]{\today}

\begin{document}

\begin{frame}
  \begin{center}
    \vspace*{\fill}
    \Huge Einführung in Ruby on Rails \\
    \vspace{0.8cm}
    \includegraphics[scale=2]{img/ruby-logo.png} \enskip
    \includegraphics[scale=0.6]{img/rails-logo.png} \\
    \vspace{0.6cm}
    \Large Johannes Gilger \& Matthias Lederhofer \\
    \small Rechen- und Kommunikationszentrum der RWTH Aachen \\
    \small Network Operation Center \\
    \vspace{1cm}
    \small 21. Juli 2010
    \vspace*{\fill}
  \end{center}
\end{frame}

\begin{frame}
  \frametitle{Übersicht}
  \begin{itemize}
    \item Ruby
    \item Rails
    \item Warum man Rails benutzen möchte
    \item Konzepte (DRY, MVC, Convention over Configuration)
    \item ActiveRecord - ORM
    \item Weiterführende Literatur, Software
  \end{itemize}
\end{frame}


\begin{frame}
  \frametitle{Der Name}
  \begin{center}
    \Huge Ruby on Rails
  \end{center}
\end{frame}

\begin{frame}
  \frametitle{Der Name}
  \begin{center}
    \Huge {\color{red}Ruby} on Rails
  \end{center}
\end{frame}

\begin{frame}
  \frametitle{Ruby}
  \begin{itemize}
    \item Objektorientierte, interpretierte Sprache
    \item Dynamisch getyped
    \item Sehr einfache Syntax
  \end{itemize}
  \begin{center}
    Ruby auch für sich betrachtet sehr interessant!
  \end{center}
\end{frame}

\begin{frame}
  \frametitle{Ruby: Klassen}
  { \tt \small
  class Customer \\
  \enskip attr\_accessor :vorname, :nachname \\
  \enskip \\
  \enskip def initialize \\
  \enskip \enskip @vorname = @nachname = '''' \\
  \enskip end \\
  \enskip \\
  \enskip def name \\
  \enskip \enskip return @vorname + '' '' + @nachname \\
  \enskip end \\
  end \\
  }
  \pause
  \begin{center}
    \small Keine Getter/Setter-Methoden schreiben:
  \end{center}
  { \tt \small
  jojo = Customer.new \\
  jojo.vorname = ''Johannes'' \\
  }
\end{frame}

\begin{frame}
  \frametitle{Ruby: Datentypen}
  { \tt \small [0, 1, 2, ''hallo'', 23.5] } - Array \\
  { \tt \small \%w(Jan, Feb, Mar, Apr)} - Stringarray \\
  { \tt \small \{ ''jojo'' => 24, ''matled'' => 25 \}} - Hash \\
  \vspace{0.3cm}
  \begin{center}
    \emph{Alle} Datentypen sind Objekte und besitzen ein Vielzahl praktischer Methoden
  \end{center}
\end{frame}

\begin{frame}
  \frametitle{Ruby: Beispiele}
  {\tt \small 2.times \{puts ''Hello''\}} \\
  {\tt \small {\color{green}=> 'Hello'}} \\
  {\tt \small {\color{green}=> 'Hello'}}
  \vspace{0.5cm}
  \pause

  {\tt \small ''RWTH Aachen''.downcase.split('''').uniq.sort.join} \\
  {\tt \small  {\color{green}=> '' acehnrtw''}} \\
  \vspace{0.5cm}
  \pause

  {\tt \small array = [1, 'hi', 3.14]} \\
  {\tt \small array.each \{|item| puts item\}} \\
  {\tt \small {\color{green}=> 1}} \\
  {\tt \small {\color{green}=> 'hi'}} \\
  {\tt \small {\color{green}=> 3.14}}
  \vspace{0.5cm}
  \pause

  {\tt \small (1..10).collect \{|x| x*x\}} \\
  {\tt \small {\color{green}=> [1, 4, 9, 16, 25, 36, 49, 64, 81, 100]}}
\end{frame}

\begin{frame}
  \frametitle{Der Name}
  \begin{center}
    \Huge Ruby {\color{red}on Rails}
  \end{center}
\end{frame}

\begin{frame}
  \frametitle{Rails}
  \begin{itemize}
    \item{Open-Source Web-Application Framework auf Grundlage von Ruby}
    \item{Aufgeteilt in mehrere Komponenten}
    \begin{itemize}
      \item{ActiveRecord - ORM (Datenbankanbindung)}
      \item{ActiveResource, ActionPack, ActiveSupport, ActionMailer}
    \end{itemize}
    \vspace{0.3cm}

    \item{2004: Veröffentlichung, seitdem viele Veränderungen}
    \item{2008: Rails-Entwicklung wird von SVN auf git umgestellt}
    \item{2010: Rails 2.3.5 ist stable, Rails läuft auch auf Ruby 1.9}
    \vspace{0.3cm}

    \item{Bekannte Projekte unter Rails:}
    \begin{itemize}
      \item{GitHub, Twitter, Lighthouse, XING, etc.}
    \end{itemize}
  \end{itemize}
\end{frame}

\begin{frame}
  \frametitle{Warum man Rails benutzen möchte}
  \begin{itemize}
    \pause
    \item{{\bf Sehr schnelle Entwicklung von Webanwendungen} \\ Insbesondere das initiale Interface}
    \pause
    \item{{\bf Unabhängigkeit von Datenbank-Engine} \\ Dadurch auch ein Sicherheitsvorteil. Manche Anwendungen können ohne ein einziges SQL-Statement geschrieben werden}
    \pause
    \item{{\bf Vereinfachung von typischen Operationen} \\ Häufig wiederholte Dinge wie Links, Bilder, \ldots}
    \pause
    \item{{\bf Vereinbarung bestimmter Konventionen} \\ Machen das Zusammenarbeiten einfacher}
    \pause
    \item{{\bf Hilfreiche Werkzeuge zum Entwickeln und Debuggen}}
  \end{itemize}
\end{frame}

\begin{frame}
  \frametitle{Convention over Configuration}
  Rails trifft bestimmte Annahmen:
  \vspace{0.5cm}
  \begin{itemize}
    \pause
    \item Datenbank: Tabellen sind pluralisiert (Customer => customers). Jede Tabelle hat einen Primärschlüssel ''id'' (INT)
    \pause
    \item Datenbank: Fremschlüssel heissen person\_id
    \pause
    \item Applikation: Aufteilung in MVC nach Ordnern, Dateien
  \end{itemize}
  \vspace{0.5cm}
  Resultat: Einschränkungen helfen. Grundlegende Entwicklungstätigkeiten werden sehr einfach.
\end{frame}

\begin{frame}
  \frametitle{MVC - Model, View, Controller}
  \begin{center}
    \includegraphics[width=7.5cm]{img/mvc.png}
  \end{center}
\end{frame}

\begin{frame}
  \frametitle{MVC - Model, View, Controller}
  \begin{center}
    \includegraphics[width=9.5cm]{img/mvc_rails.png}
  \end{center}
\end{frame}

\begin{frame}
  \frametitle{ORM - Object Relational Mapper}
  {\bf \small SQL:} \\
  {\tt \small CREATE TABLE customers (''vorname'' VARCHAR(255),} \\
  {\tt \small \enskip ''nachname'' VARCHAR(255));} \\
  \vspace{0.3cm}
  \pause
  
  {\bf \small app/models/customer.rb} \\
  {\tt \small class Customer < ActiveRecord::Base} \\
  {\tt \small \enskip def name} \\
  {\tt \small \enskip \enskip return self.vorname + '' '' + self.nachname} \\
  {\tt \small \enskip end} \\
  {\tt \small end} \\
  \vspace{0.3cm}
  \pause

  {\bf \small irb} {\small \tt (ruby script/console)} \\
  {\tt \small jojo = Customer.new(:vorname => ''Johannes'',} \\
  {\tt \small \enskip :nachname => ''Gilger'')} \\
  \vspace{0.3cm}
  \pause

  {\tt \small puts jojo.name} \\
  {\tt \small {\color{green}=> ''Johannes Gilger''}}
\end{frame}

\begin{frame}
  \frametitle{DRY - Don't Repeat Yourself}
  {\tt \small CREATE TABLE customers (''vorname'' VARCHAR(255),} \\
  {\tt \small \enskip ''nachname'' VARCHAR(255));} \\
  \vspace{0.3cm}

  {\tt \small class Customer < ActiveRecord::Base} \\
  {\tt \small \enskip def name} \\
  {\tt \small \enskip \enskip return self.vorname + '' '' + self.nachname} \\
  {\tt \small \enskip end} \\
  {\tt \small end} \\
  \vspace{0.3cm}
  \pause

  \begin{center}
  Wo kommen Customer.vorname und Customer.nachname her? \\
  \pause
  \vspace{0.1cm}
  => {\tt Customer} ist Unterklasse von {\tt ActiveRecord::Base}! \\
  \vspace{0.3cm}
  \end{center}
  Klasse: Tabelle ({\tt class Customer}) \\
  Instanz: Zeile einer Tabelle ({\tt jojo = Customer.new}) \\
  Attribut einer Instanz: Zelle ({\tt jojo.vorname = ''Johanna''})
\end{frame}

\begin{frame}
  \frametitle{DRY - Models}
  Validation in den Models, alles was in die Datenbank soll wird über das Model geleitet \\
  \tt \small
  \vspace{0.5cm}
  class Customer < ActiveRecord::Base \\
  \enskip validates\_uniqueness\_of :login, :on => :create \\
  \enskip validates\_confirmation\_of :password \\
  \enskip validates\_length\_of :login, :within => 3..40 \\
  \enskip validates\_length\_of :password, :within => 5..40 \\
  end
\end{frame}

\begin{frame}
  \frametitle{ActiveRecord - Assoziationen}
  Assoziationen machen Arbeiten mit zusammenhängenden Daten sehr einfach \\
  \tt \small
  \vspace{0.5cm}
  class Order < ActiveRecord::Base \\
  \enskip belongs\_to :customer \\
  end \\
  \vspace{0.3cm}
  class Customer < ActiveRecord::Base \\
  \enskip has\_many :orders \\
  end \\
  \vspace{0.5cm}
  ich = Customer.find\_by\_vorname(''Johannes'') \\
  ich.orders \\
  {\color{green}=> [\#order1, \#order2, \ldots]}
\end{frame}

\begin{frame}
  \frametitle{ActiveRecord - Assoziationen}
  \begin{center}
    \includegraphics[width=10.5cm]{img/has_many.png}
  \end{center}
\end{frame}

\begin{frame}
  \frametitle{ActiveRecord - Assoziationen}
  \begin{center}
    \includegraphics[width=10.5cm]{img/belongs_to.png}
  \end{center}
\end{frame}

\begin{frame}
  \frametitle{ActiveRecord - Assoziationen}
  \begin{center}
    \includegraphics[width=7.5cm]{img/habtm.png}
  \end{center}
\end{frame}

\begin{frame}
  \frametitle{Der Controller}
  Die Aufgaben des Controllers kann man unterteilen in
  \begin{enumerate}
    \item Daten aus einem oder mehreren Models holen (sprich: Instanzen)
    \item Eventuell etwas mit diesen Daten machen
    \item Daten an ein View weitergeben zur Anzeige
  \end{enumerate}
\end{frame}

\begin{frame}
  \frametitle{Trennung Controller \& View}
  Aufgaben: \\
  \begin{itemize}
    \item {\bf Controller}: Programmlogik
    \item {\bf View}: Darstellung / Layout
  \end{itemize}

  Aufbau: \\
  \begin{itemize}
    \item {\bf Controller}: Einfache Ruby-Klasse
    \item {\bf View}: eRuby, Embedded Ruby in HTML (wie z.B. PHP)
  \end{itemize}
  \pause
  Beispiel: \\
  \small \tt
  <\% 3.times do \%> \\
  \enskip <b><\%= ''Hello World'' \%></b><br/> \\
  <\% end \%>
\end{frame}

\begin{frame}
  \frametitle{Controller \& View}
  {\bf \small app/controllers/people\_controller.rb}\\
  { \tt \small
    class {\color{red}People}Controller < ApplicationController \\
    \enskip def {\color{green}list} \\
    \enskip \enskip @customers = Customer.find :all \\
    \enskip end \\
    end \\
  }
  \pause
  \vspace{0.3cm}
  {\small \bf app/views/people/list.rhtml} \\
  { \tt \small
    <table> \\
    \enskip <\% @customers.each do customer \%> \\
    \enskip \enskip <tr><td><\%= customer.vorname \%></td> \\
    \enskip \enskip \enskip <td><\%= customer.nachname \%></td></tr> \\
    \enskip <\% end \%> \\
    </table>
  }
  \pause
  \begin{center}
    Aufruf von {\tt http://localhost:3000/{\color{red}people}/{\color{green}list}}
  \end{center}
\end{frame}

\begin{frame}
  \frametitle{Rails - Anwendungsordner}
  \begin{center}
    Neue Rails-Applikation mit: {\tt rails neuesprojekt}
    \pause
    \small \tt
    \begin{columns}[t]
      \begin{column}{5cm}
        app \\
        \enskip controllers \\
        \enskip helpers \\
        \enskip models \\
        \enskip views \\
        config \\
        db \\
        doc \\
        lib \\
        log \\
        public \\
        script \\
        test \\
        tmp \\
        vendor
      \end{column}
      \pause
      \begin{column}{5cm}
        app \\
        \enskip controllers \\
        \enskip helpers \\
        \enskip models \\
        \enskip views \\
        config \\
        {\color{gray}db} \\
        {\color{lightgray}doc} \\
        {\color{lightgray}lib} \\
        {\color{gray}log} \\
        public \\
        script \\
        {\color{lightgray}test} \\
        {\color{lightgray}tmp} \\
        {\color{lightgray}vendor}
      \end{column}
    \end{columns}
  \end{center}
\end{frame}

\begin{frame}
  \frametitle{Rails - Anwendungsordner}
  \small
  {\tt app} \\
  \enskip {\tt controllers} \\ \enskip\enskip Controller, verbinden Models und Views \\
  \enskip {\tt helpers} \\ \enskip\enskip Hilfsmethoden die in mehreren Views benutzt werden \\
  \enskip {\tt models} \\ \enskip\enskip Models (entsprechend z.B. den Tabellen in der Datenbank) \\
  \enskip {\tt views} \\ \enskip\enskip HTML-Templates die zur Anzeige der Webseite benutzt werden \\
  {\tt config} \\ \enskip\enskip Anwendungs- \& Datenbankkonfiguration \\
  {\tt public} \\ \enskip\enskip Statische öffentliche Dateien (Bilder, Stylesheets) \\
  {\tt script} \\ \enskip\enskip Rails Hilfsprogramme und Debugging-Tools \\
\end{frame}

\begin{frame}
  \frametitle{Rails - Entwicklungshilfen}
  \small
  {\tt script} \\
  \enskip {\tt console} \\ \enskip\enskip Ruby-Konsole, schnelles Testen von z.B. Models\\
  \enskip {\tt dbconsole} \\ \enskip\enskip Praktischer Shortcut, User/Passwort aus {\tt config/database.yml} \\
  \enskip {\tt generate} \\ \enskip\enskip Allround-Generator für Grundgerüste (Controllers, Models, etc.) \\
  \enskip {\tt plugin} \\ \enskip\enskip Plugins installieren \\
  \enskip {\tt server} \\ \enskip\enskip Entwicklungs-Webserver auf localhost:3000 starten
\end{frame}

\begin{frame}
  \frametitle{Rails - Installation \& Maintenance}
  Ruby benutzt eigene Paketverwaltung: \emph{RubyGems} \\
  \pause
  \vspace{0.3cm}
  Vorteile:
  \begin{itemize}
    \item Schnelle und einfache Installation auch kleiner Programme
    \item Programme (''Gems'') sind unabhängig von Distribution
    \item Parallele Installation verschiedener Versionen
  \end{itemize}
  \pause
  \vspace{0.3cm}
  Befehle:
  \begin{itemize}
    \item {\tt gem list $--$local} - Installierte Gems anzeigen
    \item {\tt gem install <name>} - Gem laden und installieren
    \item {\tt gem update} - Installierte Gems updaten
  \end{itemize}
\end{frame}

\begin{frame}
  \frametitle{Rails - Installation \& Maintenance}
  Rails benutzt Version mit der das Projekt erstellt wurde. Bei einem
  Update der Gems ({\tt rails, activerecord, ...}) wird \emph{nicht} automatisch
  die neuste Version benutzt. \\
  \vspace{0.3cm}
  Rails-Applikation updaten: \\
  \vspace{0.2cm}
  In {\tt config/environment.rb} muss {\tt RAILS\_GEM\_VERSION} angepasst werden.
\end{frame}

\begin{frame}
  \frametitle{Rails - Lernen}
  \begin{itemize}
    \item Ein Rails Tutorial benutzen \\ Oftmals kleinere Anwendungen (Weblog, etc.)
    \item Bestehende Rails Projekte angucken (z.B. aus dem NOC) \\ Nachverfolgen von Webseitenaufruf über Controller, Model, View
    \item Datenbankschema ausdenken und mit Scaffold-Generator Code erzeugen
    \item {\bf Ruby} benutzen (für kleinere Tools, Parser, etc.)
  \end{itemize}
  \pause
  \vspace{0.5cm}
  {\bf Vorsicht!}\\
  Rails entwickelt sich sehr schnell und Best-Practices haben sich mehrfach geändert. Deshalb immer 
  sicher stellen dass Dokumentation auf aktuellem Stand ist.
\end{frame}

\begin{frame}
  \frametitle{Rails - Alternativen}
  Rails ist \emph{ein} Framework von vielen:
  \begin{itemize}
    \item {\bf Symfony} (PHP) \\ \url{http://www.symfony-project.org/}
    \item {\bf CakePHP} (PHP) \\ \url{http://cakephp.org/}
    \item {\bf Django} (Python) \\ \url{http://www.djangoproject.com/}
    \item {\bf Merb} (Ruby) \\ \url{http://www.merbivore.com/}
    \item {\bf Catalyst} (Perl) \\ \url{http://www.catalystframework.org/}
  \end{itemize}
  Viele neuere Frameworks verhalten sich wie Rails
\end{frame}

\begin{frame}
  \frametitle{Literatur zu Ruby \& Rails}
  \small
  \begin{itemize}
    \item {\bf Rails Guides} - \url{http://guides.rubyonrails.org/} \\ Längere Tutorials, generell und über einzelne Komponenten
    \item {\bf Agile Web Development with Rails, 3rd Edition} \\ The Pragmatic Bookshelf, ISBN: 978-1-93435-616-6 \\ Sehr ausführliches Rails-Buch mit Tutorial-Teil
    \item {\bf Rails API} - \url{http://api.rubyonrails.org/} \\ Rails API Dokumentation
    \item {\bf Railscasts} - \url{http://railscasts.com/} \\ Kurze Screencasts zu einzelnen Techniken
    \item {\bf Ruby Core rDoc} - \url{http://www.ruby-doc.org/core/} \\ Klassen- und Library-Referenz \\ \url{http://www.ruby-doc.org/docs/ProgrammingRuby/html/builtins.html} (Alt)
    \item {\bf Programming Ruby, 2nd Edition} \\ The Pragmatic Bookshelf, ISBN: 978-0-9745-1405-5 \\ Das Pickaxe-Book, sehr umfassend (mit Referenz)
  \end{itemize}
\end{frame}

\begin{frame}
  \frametitle{The End}
  \begin{center}
    \Huge Fragerunde!
    \vspace{1.5cm}

    \small Folgefragen können gerne an gilger@rz.rwth-aachen.de und lederhofer@rz.rwth-aachen.de gerichtet werden. \\
    Oder einfach vorbei schauen ;)
  \end{center}
\end{frame}

\end{document}
